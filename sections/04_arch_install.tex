\section{Install Arch Linux}
\subsection {Manually}

\subsection {Archinstall}
要安裝 Arch Linux 除了手動打指令外,現在也有個比較方便的腳本安裝,相較以往繁雜指令安裝,\code{archinstall} 提供一個方便的 TUI 介面適合快速安裝。

\subsubsection {Reflector}
不過在開始之前,必須先確保網路連線正常,詳細部分可以回去參考 Arch Wiki。接著建議設定 mirrors 可以讓之後安裝的連線速度比較快,可以使用 \code{reflector -h} 他有詳細的參數指引。其中 mirror 預設儲存路徑為 \code{/etc/pacman.d/mirrorlist},我通常會使用以下指令。
\begin{codeblock}{bash}
reflector --country Taiwan,Japan \
          --sort rate --n 10 \
          --save /etc/pacman.d/mirrorlist
\end{codeblock}
我會先用 \code{country} 找出台灣跟鄰近日本這兩個地區,接著我的排序法是用 \code{rate},因為我的目的是要找出最快的10個,如果不到10個也沒關係,\code{reflector} 會自動處理,然後將結果紀錄在 \code{/etc/pacman.d/mirrorlist}。
